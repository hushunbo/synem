\section{Discussion}
The matching functional presented in this work reduces the multi-modal problem to the mono-modal case by estimating two transfer functions between the two modalities. The main novelties introduced by our functional are: 1) it computes two global transfer functions that makes the local relationship between image modalities as close to affine as possible, and 2) it models the registration problem symmetrically, which allows us to use both transfer functions instead of selecting one of them arbitrarily. Experimental results suggest that ECC outperforms point-wise functionals such as MI and EM for the same reason CC outperforms SSD in the mono-modal case: the local linear model associated with the CC metric explains spatial inhomogeneities introduced by the bias field, which cannot be handled by the simpler SSD metric \cite{Wang2014}, while in the multi-modal case, the joint PDF and the transfer function computed by MI and EM, respectively, are unable to capture the non-stationary relationship (caused, for example, by the bias field in MR images) between both modalities. Therefore, with ECC, the global non-linear transfer functions and the local linear model complement each other. The estimation of both transfer functions is naturally introduced into the formulation of the SyN algorithm developed by Avants {\it et al.} \cite{Avants2008, Avants2011}. The SyN algorithm may be regarded as a Generalized EM algorithm (GEM) \cite{Neal1998}, in which the energy is not fully optimized at each step with respect to the transformations $\phi_{I}, \phi_{J}$, but only reduced at each iteration. It has been shown that a full optimization in the maximization step is not necessary, and a local maximum of the likelihood is still reached with partial optimizations \cite{Neal1998}. It is important to notice that according to \cite{Avants2008}, the greedy SyN algorithm achieves inverse consistency by performing explicit deformation field inversions at each iteration (lines 7, 8 of Algorithm \ref{alg:Greedy_SyN}). The inversion method used is a variation of Chen's fixed-point algorithm \cite{Chen2008}. Therefore, it is unnecessary to introduce any constraint on the local Jacobians to ensure inverse consistency. Technically, the behavior of the algorithm for deformation field inversion should be analyzed more carefully in the case that the input deformation field is not invertible. In that case, as discussed in \cite{Chen2008}, the fixed-point algorithm may still converge to a deformation field that ensures inverse consistency. However, the discussion on the transformation model is beyond the scope of this work.\\

The proposed validation protocol using semi-synthetic images \cite{Ocegueda2015} allows us to quantitatively assess the accuracy of multi-modal registration algorithms under realistic conditions for three different image modalities and their combinations: T1, T2 and PD. The full validation required 2,754 registrations for each of the 4 methods under evaluation (a total of 11,016 registrations). However, the lack of annotated data sets containing T1 and diffusion images from the same subjects did not allow us to quantitatively validate our matching functional for $B_{0}$-T1 co-registration. Still, the proposed indirect quantitative validation (containing only real data) allows us to assess the registration of $B_{0}$ and $T1$ images (although not from the same subject).

\section{Conclusion}
We presented a new matching functional, which we call Expected Cross Correlation (ECC), for multi-modal image registration. We evaluated our matching functional using the publicly available IBSR database following the methodology of Klein {\it et al.} \cite{Klein2009, Klein2010} and Rohlfing \cite{Rohlfing2012} to show that it is competitive for mono-modal image registration. To evaluate the performance of ECC for multi-modal registration in a realistic, controlled experiment, we used the Brainweb \cite{Cocosco1997, Kwan1999} template to generate synthetic T2 and PD images for each of the IBSR T1 volumes. Experiments show that the CC metric is dramatically affected by the change of modality while the performance of ECC is less affected, remaining comparable to the mono-modal case and comparing favourably to the reference implementation of SyN with MI provided by the ANTS software. Our algorithms are publicly available in DIPY \cite{Garyfallidis2014}.


%Still, we did not exploit the full potential of this validation protocol: it is also possible to study the performance of the same multi-modal registration algorithms in the presence of noise and spatial inhomogeneities, since the Brainweb template allows us to realistically simulate these distortions.
%It is important to note that, even though the proposed matching functional compared favourably against the other functionals under test for $B_{0}$-T1 registration, that does not mean that it is accurate enough to be used for correction of susceptibility induced geometric distortions, since quantitative evaluation only measures accuracy over relatively large anatomical regions.\\
