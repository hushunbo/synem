\section{Discussion}
The matching functional presented in this work reduces the multi-modal problem to the mono-modal case by estimating two transfer functions between the two modalities. \textcolor{red}{In summary, the reasoning is simply to use the CC metric to account for inaccuracies in the transfer function (in the multimodal case) the same way that CC is used to add robustness to local brightness and contrast variations in the mono-modal case (see for example Evangelidis {\it et al.} \cite{Evangelidis2008} and Wang and Pan \cite{Wang2014} for further discussion).} Experimental results suggest that ECC outperforms point-wise functionals such as MI, EM and BFP. The estimation of both transfer functions is naturally introduced into the formulation of the SyN algorithm developed by Avants {\it et al.} \cite{Avants2008, Avants2011}. The proposed validation protocol using semi-synthetic images \cite{Ocegueda2015} allows us to quantitatively assess the accuracy of multi-modal registration algorithms under realistic conditions for three different image modalities and their combinations: T1, T2 and PD. The full validation required \textcolor{red}{2,754} registrations for each of the \textcolor{red}{5} methods under evaluation (a total of \textcolor{red}{13,770} registrations). However, the lack of annotated data sets containing T1 and diffusion images from the same subjects did not allow us to quantitatively validate our matching functional for $B_{0}$-T1 co-registration. Still, the proposed indirect quantitative validation (containing only real data) allows us to assess the registration of $B_{0}$ and $T1$ images (although not from the same subject). \textcolor{red}{The strongest motivation of this study is  the correction of susceptibility-induced geometric distortions using a non-EPI undistorted image (possibly of different modality) as reference when no more than one EPI image is available. Specific issues towards this goal that will be addressed in future research are:
\begin{itemize}
    \item{The transformation must be constrained along the phase-encode direction of the EPI image. This means that the following point is crucial:}
    \item{Motion between the EPI image and the non-EPI reference must be accurately estimated. Slight misalignment would produce spurious deformations which would be incorrectly interpreted as being produced by the off-resonance field.}
    \item{The relationship between the transfer function and the Jacobian modulation needs to be carefully defined. The ideas of Bhushan {\it et al.} \cite{Bhushan2015} would be a good starting point.}
	\item{All results shown in this work were obtained using brain extraction as preprocessing to improve the transfer function estimation (this is a standard pre-processing, e.g. \cite{Bhushan2015}). Brain extraction is a challenging problem especially in the presence of strong attenuation along expanding areas in EPI which makes brain extraction a semi-automatic process in practice.}
\end{itemize}}

%The SyN algorithm may be regarded as a Generalized EM algorithm (GEM) \cite{Neal1998}, in which the energy is not fully optimized at each step with respect to the transformations $\phi_{I}, \phi_{J}$, but only reduced at each iteration. It has been shown that a full optimization in the maximization step is not necessary, and a local maximum of the likelihood is still reached with partial optimizations \cite{Neal1998}. 

%It is important to notice that according to \cite{Avants2008}, the greedy SyN algorithm achieves inverse consistency by performing explicit deformation field inversions at each iteration (lines 7, 8 of Algorithm \ref{alg:Greedy_SyN}). The inversion method used is a variation of Chen's fixed-point algorithm \cite{Chen2008}. Therefore, it is unnecessary to introduce any constraint on the local Jacobians to ensure inverse consistency. Technically, the behavior of the algorithm for deformation field inversion should be analyzed more carefully in the case that the input deformation field is not invertible. In that case, as discussed in \cite{Chen2008}, the fixed-point algorithm may still converge to a deformation field that ensures inverse consistency. However, the discussion on the transformation model is beyond the scope of this work.\\



\section{Conclusion}
We presented a new matching functional, which we call Expected Cross Correlation (ECC), for multi-modal image registration. We evaluated our matching functional using the publicly available IBSR database following the methodology of Klein {\it et al.} \cite{Klein2009, Klein2010} and Rohlfing \cite{Rohlfing2012} to show that it is competitive for mono-modal image registration. To evaluate the accuracy of ECC for multi-modal registration in a realistic, controlled experiment, we used the Brainweb \cite{Cocosco1997, Kwan1999} template to generate synthetic T2 and PD images for each of the IBSR T1 volumes. Experiments show that the CC metric is dramatically affected by the change of modality while the accuracy of ECC is less affected, remaining comparable to the mono-modal case and comparing favourably to the reference implementation of SyN with MI \cite{Mattes2003} provided by the ANTS software \textcolor{red}{ and in-house implementations of EM \cite{Arce-santana2014} and BFP \cite{Guimond2001}}. Our algorithms are publicly available in DIPY \cite{Garyfallidis2014}.


%Still, we did not exploit the full potential of this validation protocol: it is also possible to study the performance of the same multi-modal registration algorithms in the presence of noise and spatial inhomogeneities, since the Brainweb template allows us to realistically simulate these distortions.
%It is important to note that, even though the proposed matching functional compared favourably against the other functionals under test for $B_{0}$-T1 registration, that does not mean that it is accurate enough to be used for correction of susceptibility induced geometric distortions, since quantitative evaluation only measures accuracy over relatively large anatomical regions.\\
