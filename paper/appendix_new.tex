\clearpage
\appendix
\numberwithin{equation}{subsection}
\section*{Appendix}
\label{sec:appendix}

\renewcommand{\thesubsection}{\Alph{subsection}}

\subsection{Gradient of the CC metric}\label{ap:CC_gradient}
We are interested in computing the gradient of
\begin{equation}
    CC(\bar{I}, \bar{J}, x) = \frac{<\bar{I}, \bar{J}>^{2}}{<\bar{I}><\bar{J}>}
\end{equation}
where the inner product is taken over an $n^{D}$ window \citep[see][eq. 4]{Avants2008}. Since the windows are considered discrete, a more precise notation
for this expression is:
\begin{equation}\label{eq:CC_definition}
    CC(y;\phi_{I}, \phi_{J}) = \frac{\left[\sum_{z\in W_{y}} \left(\tilde{I}(z) - \mu_{y}\right)\left(\tilde{J}(z) - \nu_{y}\right)\right]^{2}}
    {\left[\sum_{z \in W_{y}}\left(\tilde{I}(z) - \mu_{y}\right)^{2}\right] \left[\sum_{z \in W_{y}}\left(\tilde{J}(z) - \nu_{y}\right)^{2}\right]} = \frac{A_{y}^{2}}{B_{y}C_{y}}
\end{equation}
where $\tilde{I}(z) = I(\phi_{I}^{-1}(z))$, $\tilde{J}(z) = J(\phi_{J}^{-1}(z))$ and the full energy is given by
\begin{equation}
    CC(\phi_{I}, \phi_{J}) = \sum_{y\in\Omega} CC(y; \phi_{I}, \phi_{J})
\end{equation}
where $W_{y}$ is the window of side $n$ centered at voxel $y$, $|W_{y}|$ is the number of voxels in window $W_{y}$ and:
\begin{equation}
    \begin{array}{lll}
        \mu_{y} &=& \frac{1}{|W_{y}|}\sum_{z \in W_{y}}\tilde{I}(z)\\[+2mm]
        \nu_{y} &=& \frac{1}{|W_{y}|}\sum_{z \in W_{y}}\tilde{J}(z)\\
    \end{array}.
\end{equation}

We wish to compute the gradient of $CC(\phi_{I}, \phi_{J})$ with respect to $\phi^{-1}_{J}(x)$, $x\in\Omega$. The set of window terms $CC(y;\phi_{I}, \phi_{J})$
that depend on $\phi^{-1}_{J}(x)$ is precisely the set of all windows $W_{y}$ such that $y \in W_{x}$. Therefore:
\begin{equation}
    \frac{\partial CC (\phi_{I}, \phi_{J})}{\partial \phi^{-1}_{J}(x)} = \sum_{y \in W_{x}} \frac{\partial CC (y; \phi_{I}, \phi_{J})}{\partial \phi^{-1}_{J}(x)}
\end{equation}
and
\begin{equation}
    \frac{\partial CC (y; \phi_{I}, \phi_{J})}{\partial \phi^{-1}_{J}(x)} =
        \frac{\left(2A_{y} B_{y}C_{y}\right)\frac{\partial A_{y}}{\partial \phi^{-1}_{J}(x)} - \left(A_{y}^{2}B_{y}\right)\frac{\partial C_{y}}{\partial \phi^{-1}_{J}(x)}}
             {B_{y}^{2} C_{y}^{2}}
\end{equation}
where
\begin{equation}
    \begin{array}{lll}
        \frac{\partial A_{y}}{\partial \phi^{-1}_{J}(x)} &=& (\tilde{I}(x) - \mu_{y})\nabla \tilde{J}(x)\\[+3mm]
        \frac{\partial C_{y}}{\partial \phi^{-1}_{J}(x)} &=& 2(\tilde{J}(x) - \nu_{y})\nabla \tilde{J}(x)
    \end{array}.
\end{equation}
After simplifying, we get
\begin{equation}\label{eq:CC_gradient}
    \frac{\partial CC (\phi_{I}, \phi_{J})}{\partial \phi^{-1}_{J}(x)} = \sum_{y \in W_{x}}
         \frac{2A_{y}}
              {B_{y}C_{y}}\left[ (\tilde{I}(x) - \mu_{y}) - \frac{A_{y}}{C_{y}}\left(\tilde{J}(x) - \nu_{y}\right)\right]\nabla \tilde{J}(x).
\end{equation}
If we define the scalar functions
\begin{equation}
    \begin{array}{lll}
        S_{a}(x) &=& \sum_{y \in W_{x}} \frac{2A_{y}}{B_{y}C_{y}}\\[+2mm]
        S_{b}(x) &=& \sum_{y \in W_{x}} \frac{2A_{y}^{2}}{B_{y}C_{y}^{2}}\\[+2mm]
        S_{c}(x) &=& \sum_{y \in W_{x}} \frac{2A_{y}}{B_{y}C_{y}} \left[ \mu_{y} - \frac{A_{y}}{C_{y}}\nu_{y}\right].
    \end{array}
\end{equation}
we can write the gradient of $CC$ with respect to $\phi^{-1}_{J}$ as:

\begin{equation}
    \psi_{J} = \nabla_{\phi^{-1}_{J}} CC(\phi_{I}, \phi_{J}) = \left[S_{a} \tilde{I} - S_{b}\tilde{J} - S_{c}\right]\nabla \tilde{J}.
\end{equation}

It is important to notice the similarity and differences between equation \eqref{eq:CC_gradient} and the expressions derived by \cite{Hermosillo2004}
and \cite{Avants2008}. On the one hand, \cite{Hermosillo2004} considered separately the cases of local and global intensity comparisons: the global comparison corresponds to setting $W_{x} = \Omega$ (summing over the full domain). The local intensity comparison is accomplished by using a Gaussian kernel centered at each location $x\in\Omega$. The limitation of using Gaussian kernels is the computational cost, as pointed out by \cite{Hermosillo2004}, which makes it necessary to perform some simplifications and use parallel computing, while using rectangular windows results in a very efficient and robust implementation, as shown by \cite{Avants2008}. On the other hand, even though the formulation of \cite{Avants2008} is the same as eq. \eqref{eq:CC_definition}, only the central voxel contributes to the sum in eq. \eqref{eq:CC_gradient} \citep[see][eqs. 6, 7]{Avants2008}. In practice, we have not observed significant differences between the quantitative results using eq. \eqref{eq:CC_gradient} of this appendix, and eqs. (6) and (7) from \cite{Avants2008}, thus we opted for using rectangular windows and the gradient computations as in \cite{Avants2008}.

\subsection{Algorithms}\label{ap:Algorithms}
\begin{algorithm}[h!]
\caption{Greedy SyN. This algorihtm was the method used for evaluating ANTS \citep{Avants2011} in the large comparative studies developed by \cite{Klein2009, Klein2010} in which it consistently ranked first.}\label{alg:Greedy_SyN}
\begin{algorithmic}[1]
\REQUIRE Gaussian kernel parameter $\sigma>0$
\REQUIRE Step size $\epsilon>0$
\REQUIRE Maximum number of iterations $T>0$
\STATE Initialize: $\phi_{i}(\cdot, 0.5) = Id, i=1, 2$
\STATE $t=0$
\REPEAT
    \STATE Warp $\tilde{I}  = I \circ \phi_{1}^{-1}(\cdot, 0.5), \tilde{J} = J \circ \phi_{2}^{-1}(\cdot, 0.5)$
    \STATE Compute the gradients $\mathbf{u}_{i} = \nabla_{\phi_{i}} \Pi(\tilde{I}, \tilde{J}), i=1,2$
%    \STATE Update $\phi_{i}(\cdot, 0.5), i=1, 2$ according to eq. \eqref{eq:gsyn_update}
    \STATE Update $\phi_{i}(\cdot, 0.5) = \phi_{i}(\cdot, 0.5) - \left( \epsilon K_{\sigma} \ast \mathbf{u}_{i} \right) \circ \phi_{i}(\cdot, 0.5)$
    \STATE Invert $\phi_{i}^{-1}(\cdot, 0.5) = invert (\phi_{i}(\cdot, 0.5)), i=1, 2$
    \STATE Invert $\phi_{i}(\cdot, 0.5) = invert (\phi_{i}^{-1}(\cdot, 0.5)), i=1, 2$
    \STATE t = t + 1
\UNTIL{$t\geq T$ or convergence}
\RETURN $\phi_{i}(\cdot, 0.5), i=1,2$
\end{algorithmic}
\end{algorithm}


\begin{algorithm}[h!]
\caption{SyN-EM. This algorithm uses a symmetric extension of the EM metric proposed by \cite{Arce-santana2014}.}\label{alg:SyNEM}
\begin{algorithmic}[1]
\REQUIRE Gaussian kernel parameter $\sigma>0$
\REQUIRE Step size $\epsilon>0$
\REQUIRE Maximum number of iterations $T>0$
\STATE Initialize: $\phi_{I} = Id, \phi_{J} = Id$
\STATE $t=0$
\REPEAT
    \STATE Warp $\tilde{I}  = I \circ \phi_{I}^{-1}, \tilde{J} = J \circ \phi_{J}^{-1}$
    \STATE E-Step (eq. \eqref{eq:expectation_transfer}) $\overline{Y}(x) = \mathbbm{E}\left[\left.\mathbf{I}\right| \mathbf{J}= J(\phi_{J}^{-1}(x))\right]$
    \STATE E-Step $\overline{Z}(x) = \mathbbm{E}\left[\left.\mathbf{J}\right| \mathbf{I}= I(\phi_{I}^{-1}(x))\right]$
    \STATE E-Step (eq. \eqref{eq:expectation_variance}) $\sigma^{2}_{Y}(x) = Var\left[\left.\mathbf{I}\right| \mathbf{J}= J(\phi_{J}^{-1}(x))\right]$
    \STATE E-Step $\sigma^{2}_{Z}(x) = Var\left[\left.\mathbf{J}\right| \mathbf{I}= I(\phi_{I}^{-1}(x))\right]$
    \STATE M-Step (eq. \eqref{eq:euler_lagrange_step1}) $\mathbf{u}_{I} = \frac{\overline{Y}(x) - \tilde{I}(x)}{||\nabla \tilde{I}(x)||^{2} + \frac{\sigma_{Y}^{2}(x)}{\tau}}\nabla \tilde{I}(x)$
    \STATE M-Step $\mathbf{u}_{J} = \frac{\overline{Z}(x) - \tilde{J}(x)}{||\nabla \tilde{J}(x)||^{2} + \frac{\sigma_{Z}^{2}(x)}{\tau}}\nabla \tilde{J}(x)$
    \STATE Update $\phi_{I} = \phi_{I} - \left(\epsilon K_{\sigma} \ast \mathbf{u}_{I} \right)\circ \phi_{I}$
    \STATE Update $\phi_{J} = \phi_{J} - \left(\epsilon K_{\sigma} \ast \mathbf{u}_{J} \right)\circ \phi_{J}$
    \STATE Invert $\phi_{I}^{-1}, \phi_{J}^{-1} = invert(\phi_{I}), invert(\phi_{J})$
    \STATE Invert $\phi_{I}, \phi_{J} = invert(\phi_{I}^{-1}), invert(\phi_{J}^{-1})$
    \STATE t = t + 1
\UNTIL{$t\geq T$ or convergence}
\RETURN $\phi_{I}, \phi_{J}$
\end{algorithmic}
\end{algorithm}


\begin{algorithm}[h!]
\caption{SyN-ECC. This algorithm uses an extension of the CC metric for multi-modal images. It uses a (global) estimation of the transfer functions between the two image modalities and measures the similarity of (local) image windows with the CC metric.}\label{alg:SyNECC}
\begin{algorithmic}[1]
\REQUIRE Gaussian kernel parameter $\sigma>0$
\REQUIRE Step size $\epsilon>0$
\REQUIRE Maximum number of iterations $T>0$
\STATE Initialize: $\phi_{I} = Id, \phi_{J} = Id$
\STATE $t=0$
\REPEAT
    \STATE Warp $\tilde{I}  = I \circ \phi_{I}^{-1}, \tilde{J} = J \circ \phi_{J}^{-1}$
    \STATE E-Step (eq. \eqref{eq:expectation_transfer}) $\overline{Y}(x) = \mathbbm{E}\left[\left.\mathbf{I}\right| \mathbf{J}= J(\phi_{J}^{-1}(x))\right]$
    \STATE E-Step $\overline{Z}(x) = \mathbbm{E}\left[\left.\mathbf{J}\right| \mathbf{I}= I(\phi_{I}^{-1}(x))\right]$
    \STATE E-Step (eq. \eqref{eq:expectation_variance}) $\sigma^{2}_{Y}(x) = Var\left[\left.\mathbf{I}\right| \mathbf{J}= J(\phi_{J}^{-1}(x))\right]$
    \STATE E-Step $\sigma^{2}_{Z}(x) = Var\left[\left.\mathbf{J}\right| \mathbf{I}= I(\phi_{I}^{-1}(x))\right]$
    \STATE M-Step (gradient of eq. \eqref{eq:ecc_metric}) $\mathbf{u}_{I} = - \nabla_{\phi^{-1}_{I}} ECC(\tilde{I}, \tilde{J} | \phi_{I}, \phi_{J})$
    \STATE M-Step $\mathbf{u}_{J} = - \nabla_{\phi^{-1}_{J}} ECC(\tilde{I}, \tilde{J} | \phi_{I}, \phi_{J})$
    \STATE Update $\phi_{I} = \phi_{I} - \left(\epsilon K_{\sigma} \ast \mathbf{u}_{I} \right)\circ \phi_{I}$
    \STATE Update $\phi_{J} = \phi_{J} - \left(\epsilon K_{\sigma} \ast \mathbf{u}_{J} \right)\circ \phi_{J}$
    \STATE Invert $\phi_{I}^{-1}, \phi_{J}^{-1} = invert(\phi_{I}), invert(\phi_{J})$
    \STATE Invert $\phi_{I}, \phi_{J} = invert(\phi_{I}^{-1}), invert(\phi_{J}^{-1})$
    \STATE t = t + 1
\UNTIL{$t\geq T$ or convergence}
\RETURN $\phi_{I}, \phi_{J}$
\end{algorithmic}
\end{algorithm} 